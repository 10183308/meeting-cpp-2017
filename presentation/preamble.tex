% !TEX root = presentation.tex
% Presentation Preamble

%%%%%%%%%%%%%%%%%%%%%%%%%%%%%%%%%%%%%%%%%%%%%%%%%%%%%%%%%%%%%%%%%%%%%%%%%%
%   BEAMER CLASS
%%%%%%%%%%%%%%%%%%%%%%%%%%%%%%%%%%%%%%%%%%%%%%%%%%%%%%%%%%%%%%%%%%%%%%%%%%

\documentclass[xcolor={dvipsnames}]{beamer}

\definecolor{llvmblue}{RGB}{60, 66, 80}

\usetheme{default}
\usecolortheme[named=orange]{structure}
\setbeamertemplate{sections/subsections in toc}[sections numbered]
\setbeamertemplate{items}[default]
\usefonttheme[onlymath]{serif}
\setbeamertemplate{caption}{\raggedright\insertcaption\par}

%%%%%%%%%%%%%%%%%%%%%%%%%%%%%%%%%%%%%%%%%%%%%%%%%%%%%%%%%%%%%%%%%%%%%%%%%%
% LANGUAGE
%%%%%%%%%%%%%%%%%%%%%%%%%%%%%%%%%%%%%%%%%%%%%%%%%%%%%%%%%%%%%%%%%%%%%%%%%%

\usepackage{CJKutf8}
\usepackage[utf8]{inputenc}
\usepackage[american]{babel}

%%%%%%%%%%%%%%%%%%%%%%%%%%%%%%%%%%%%%%%%%%%%%%%%%%%%%%%%%%%%%%%%%%%%%%%%%%
% 	GENERAL PACKAGES
%%%%%%%%%%%%%%%%%%%%%%%%%%%%%%%%%%%%%%%%%%%%%%%%%%%%%%%%%%%%%%%%%%%%%%%%%%

\usepackage{hyperref}
\usepackage{graphicx}
\usepackage{xcolor}
\usepackage{textcomp}
\usepackage{anyfontsize}
\usepackage{cancel}
\usepackage{tcolorbox}

\usepackage{ulem}
\renewcommand{\ULthickness}{.6pt}%

%%%%%%%%%%%%%%%%%%%%%%%%%%%%%%%%%%%%%%%%%%%%%%%%%%%%%%%%%%%%%%%%%%%%%%%%%%
% 	MATH
%%%%%%%%%%%%%%%%%%%%%%%%%%%%%%%%%%%%%%%%%%%%%%%%%%%%%%%%%%%%%%%%%%%%%%%%%%

\usepackage{amsmath}
\usepackage{blkarray}
\usepackage{gensymb}
\usepackage{relsize}
\usepackage{cancel}
\usepackage{amssymb}
\usepackage{latexsym}

%%%%%%%%%%%%%%%%%%%%%%%%%%%%%%%%%%%%%%%%%%%%%%%%%%%%%%%%%%%%%%%%%%%%%%%%%%
% 	DRAWING
%%%%%%%%%%%%%%%%%%%%%%%%%%%%%%%%%%%%%%%%%%%%%%%%%%%%%%%%%%%%%%%%%%%%%%%%%%

\usepackage{tikz}
\usepackage{pgfplots}

\pgfplotsset{compat=1.3}
\usetikzlibrary{3d}
\usetikzlibrary{shapes}
\usetikzlibrary{3d}
\usepgflibrary{fpu}

%%%%%%%%%%%%%%%%%%%%%%%%%%%%%%%%%%%%%%%%%%%%%%%%%%%%%%%%%%%%%%%%%%%%%%%%%%
% 	CODE
%%%%%%%%%%%%%%%%%%%%%%%%%%%%%%%%%%%%%%%%%%%%%%%%%%%%%%%%%%%%%%%%%%%%%%%%%%

\usepackage{algorithm}
\usepackage[noend]{algpseudocode}
\usepackage{listings}

\definecolor{stringgreen}{RGB}{50,220,15}
\definecolor{brightred}{RGB}{234, 38, 49}

\lstset{%
  inputencoding=utf8,
	basicstyle=\ttfamily\footnotesize,
	language=C++,
	aboveskip=3mm,
	belowskip=3mm,
	showstringspaces=false,
	columns=flexible,
	numbers=none,
	numberstyle=\tiny\color{red},
	keywordstyle=\color{magenta},
	commentstyle=\color{gray},
	stringstyle=\color{stringgreen},
  emphstyle=\color{brightred},
	breaklines=true,
	breakatwhitespace=true,
	tabsize=2,
  escapeinside={@}{@},
  emph={\#include},
  alsoletter={\#}
}

%%%%%%%%%%%%%%%%%%%%%%%%%%%%%%%%%%%%%%%%%%%%%%%%%%%%%%%%%%%%%%%%%%%%%%%%%%
% 	CONFIG
%%%%%%%%%%%%%%%%%%%%%%%%%%%%%%%%%%%%%%%%%%%%%%%%%%%%%%%%%%%%%%%%%%%%%%%%%%

\graphicspath{{figures/}}
% \addtolength{\parskip}{\baselineskip}

%%%%%%%%%%%%%%%%%%%%%%%%%%%%%%%%%%%%%%%%%%%%%%%%%%%%%%%%%%%%%%%%%%%%%%%%%%
% 	BEAMER FOOTER
%%%%%%%%%%%%%%%%%%%%%%%%%%%%%%%%%%%%%%%%%%%%%%%%%%%%%%%%%%%%%%%%%%%%%%%%%%

\setbeamertemplate{footline}[text line]{%
  \parbox{\linewidth}{%
    \vspace*{-15pt}%
    \scriptsize
    Peter Goldsborough\hspace{2cm}%
    {\small\texttt{Deep Learning w/ C++}}\hfill%
    \insertpagenumber}%
}
\setbeamertemplate{navigation symbols}{}

%%%%%%%%%%%%%%%%%%%%%%%%%%%%%%%%%%%%%%%%%%%%%%%%%%%%%%%%%%%%%%%%%%%%%%%%%%
% 	NEW COMMANDS
%%%%%%%%%%%%%%%%%%%%%%%%%%%%%%%%%%%%%%%%%%%%%%%%%%%%%%%%%%%%%%%%%%%%%%%%%%

\newcommand{\hs}[1]{\hspace{#1}}
\newcommand{\vs}[1]{\vspace{#1}}

\newcommand{\red}{\color{red}}

\newcommand{\inlineitem}[2]{{\color{orange}#1.} #2 \hspace{0.5cm}}

\newcommand{\textframe}[2]{%
  \begin{slide}{#1}
    {
      \fontsize{48}{48}\selectfont
      \color{llvmblue}
      #2
    }
  \end{slide}
}

\newcommand\invisiblesection[1]{%
  \refstepcounter{section}%
  \addcontentsline{toc}{section}{\protect\numberline{\thesection}#1}%
  \sectionmark{#1}
}

\newcommand{\pitem}{\pause\item}

\newcommand{\frameheader}[1]{{\Large #1}\vspace{0.5cm}}

\newcommand{\random}[1]{\pdfuniformdeviate #1}

\newcommand{\randomcolor}{%
  \definecolor{randomcolor}{RGB}
   {
    \pdfuniformdeviate 256,
    \pdfuniformdeviate 256,
    \pdfuniformdeviate 256
   }%
}

\newcommand{\randomgray}{%
  \newcount\gray\relax
  \gray=\pdfuniformdeviate 256\relax
  \definecolor{randomgray}{RGB}
   {
    \the\gray,
    \the\gray,
    \the\gray
   }%
}

\newcommand{\numbersquare}[3]{
  \draw #1 rectangle ++(#2, #2) node [midway, black] {#3};
}

\newcommand{\colornumbersquare}[4]{
  \draw [#4] #1 rectangle ++(#2, #2) node [midway] {#3};
}

\newcommand{\onesquare}[2]{ \numbersquare{#1}{1}{#2} }

%%%%%%%%%%%%%%%%%%%%%%%%%%%%%%%%%%%%%%%%%%%%%%%%%%%%%%%%%%%%%%%%%%%%%%%%%%
% 	NEW ENVIRONMENTS
%%%%%%%%%%%%%%%%%%%%%%%%%%%%%%%%%%%%%%%%%%%%%%%%%%%%%%%%%%%%%%%%%%%%%%%%%%

% http://tex.stackexchange.com/questions/78462/labelling-ax-b-under-an-actual-matrix
\newenvironment{sbmatrix}[1]
 {\def\mysubscript{#1}\mathop\bgroup\begin{bmatrix}}
 {\end{bmatrix}\egroup_{\textstyle\mathstrut\mysubscript}}

\newenvironment{spmatrix}[1]
{\def\mysubscript{#1}\mathop\bgroup\begin{pmatrix}}
{\end{pmatrix}\egroup_{\textstyle\mathstrut\mysubscript}}

\newenvironment{slide}[1]
{
  \centering
	\section{#1}
	\begin{frame}
		\frametitle{#1}
}
{
	\end{frame}
}

\newenvironment{fslide}[1]
{
  \centering
	\section{#1}
	\begin{frame}[fragile]
		\frametitle{#1}
}
{
	\end{frame}
}
